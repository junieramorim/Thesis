******************************
deepL.com (abstract em portugues)
Verificar o uso do termo processo no texto (Process, model, what???)


Processo, artefato ou recurso?
Um modelo especifico? Uma atividade de gerar um modelo?
Dado as limitações existentes e QA proporemos um processo (QA specification, model construction, ...)
O processo possui atividades a serem executadas

Channel system parametrizado definindo processo com um conjunto de etapas a serem realizadas. As etapas do processo podem estar relacionadas com os passos da Fig 4.7 (QA definition)



Related Work:
Iniciar explicando a estrutura e seguir uma logica do texto.
pode ser quebrado em subtopicos
C2, DSPL, channel system


******************************

Tornar explicito o TeamWork - Potential recognition and team formation


%Referencing to collaborative work between agents, the MPL dimension represents the second step of a teamwork described by \textit{Keplicz et al.}~\cite{Dunin-Keplicz2010} and called \textit{team formation}. When defining an organization structure of the members, i.e., DSPL, we are selecting a team to perform a specific set of tasks.



**********************************************************

Confirmações apresentando evidências (formal ou empiricas) e utilizando as definições PRECISAS que foram feitas anteriormente
>>>>>> Conceitos utilizados no texto precisam estar DEFINIDOS previamente (de maneira clara)

Empíricas: exemplos, experimentos, referencias
Formal:

Obs Forma.: 1 paragrafo topico e demais paragrafos fazendo o unfold e se relacionando.


******************************
INTRODUCAO

CONTEXTO: C2 E AGILIDADE EM C2
    O QUE É C2
    IMPORTANCIA DE C2
    EXEMPLO DE C2
    O QUE É AGILIDADE

PROBLEMA: QUAIS OS GAPS





%Let's consider the entity ($e$) that represent the status of this member. The sensors onboard are able to realize all new environment conditions and, with these results, to perform adaptation in itself. This situation represents the capacity to monitor changes in the member and in the environment. Additionally, there is a list of tasks ($T$) to be performed that characterize the mission. Changes in this list represents mission modification. More precisely, with these two elements it is possible to represent changes in circumstances that causes a system adaptation either in member level or in coordination level through C2 Approach application. 







%The element $e$, represented by its feature model ($FM$), assumes a valid configuration ($c \in \llbracket FM \rrbracket$) to be able to perform set of tasks ($T$) allocated to it. To extend current studies that does not explore communication and coordination between DSPLs, the dimension C2 Approach of the MDSPL defines how the elements share awareness and data. The communication structure and protocol will define how the elements of a team exchange information about their configuration and collaboration. This dialogue follows a flow that defines how the entities are organized inside the team or among the teams.
