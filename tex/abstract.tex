Command and Control (C2), in its historical origin, is linked to the application of classic military strategies where there was a single centralized command and an inflexible chain of command between the elements that composed the acting forces. C2 is not an end in itself, but a process whose goal is to optimize the application of resources in order to accomplish a given mission. However, in a modern C2 context, the dynamism of the mission, the team and the environment is a necessary assumption and, thus, the organization of the team to accomplish a mission becomes a challenge requiring constant adaptations. This ability to adapt to new circumstances characterizes the C2 Agility. However, current studies do not show the impacts in this capacity of agility caused by the choices of C2 strategy, represented by the level of information spread, by the organization of the performers and by the capacity of decision making. In addition, recent works do not consider the measurement of Quality Attributes (QA), which makes the models and simulations poorly adherent to the reality of missions, where at least the cost can be an obstacle to their achievement. To address these issues, our goal is to apply concepts of Self-Adaptive Systems (SAS) with an approach using Dynamic Software Product Lines (DSPL) to represent the elements that make up the C2 System and that are organized into teams. Working with the concepts of configuration and coordination, we propose a process to be applied in C2 that provides the generation of models that seek to ensure agility of C2. These models provide for the appropriate choice of the C2 strategy for the correct coordination of members, combined with the ability to reconfigure the executors in order to ensure agility to face the changes in circumstances that may occur. To evaluate the proposed solution, we plan to apply questionnaires and workshops to C2 domain experts, in addition to conducting simulations that show that the hypotheses outlined are correct.