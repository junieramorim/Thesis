The definition of C2 originally referred to the classical military strategies applied in the XIX and XX centuries, with a central commander and an inflexible hierarchical chain of command~\cite{Alberts2006}. However, this concept is totally incompatible with the modern warfare and environment, where information sharing causes impacts over the results.

The \gls{DoD}~\cite{dod01} defines C2 as the exercise of authority and direction by a properly designated commander over assigned and attached forces in the accomplishment of the mission. However, this definition does not permit to identify the existence of C2 in the organization and to perform its assessment. Furthermore, it is limited in terms of research aspects and does not involves many other C2 applications in all possible contexts. It is more compatible when we have a static scenario where changes do not occur. In this case, the entities involved have predictable behavior and it is not required to have a mechanism to do a constant monitoring. 

NATO extended this definition to the functions of commanders, staffs, and other Command and Control bodies in maintaining the combat readiness of their forces, preparing operations, and directing troops in the performance of their tasks~\cite{FRANCE2014}. With this, C2 becomes a concept that naturally involves a dynamic context because it empowers the system with the capacity to deal with changes in circumstances~\cite{Power01}.

According to \textit{Stanton et al.}~\cite{Stanton2007} and \textit{Mason et al.}~\cite{Mason2001}, C2 relies on information and awareness sharing rather than simply data broadcast. Information needs to respect the rules and logical conditions to reach the right target and to cause the right effect. This information sharing occurs on a network structure that satisfies the context requirements, including mission details and context restrictions. 

Empower the individuals on the edge of the organization, spreading the information in the right way and compatible quantity, is the new challenge to support a wider C2 definition. \textit{Alberts and Hayes} in \cite{Power01} presented the term \textit{Power to the Edge} aiming to expose the idea of take the information to all elements in the organization. It is directly related to the information and awareness spread level.

The same work~\cite{Power01} presented the three key dimensions of C2: Allocation of Decision Rights, Patterns of Interaction and Distribution of Information. These dimensions, not completely independent, compose the C2 Approach Space described by \textit{Alberts et al.} in \cite{Alberts2006} show in Figure \ref{c2s}.

\figura[!h]{C2_Space}{C2 Approach Space (from \cite{FRANCE2014})}{c2s}{width=0.6\textwidth}%

\textbf{a) Allocation of Decison Rights (ADR)}

Decision rights are the capability of some individuals to make choices related to a specific topic. The allocation of this capability aims the unity of purpose. With the ADR more distributed, the leadership changes according to the information shared.

The ADR axis represents, in an endpoint, the total centralization of decision rights, and in the other end of the spectrum, the total capability to take decisions for all members. In this case, no leadership is identified.

\textbf{b) Patterns of Interaction (PoI)}

Basically, PoI defines the network topology used by the entities. According to \textit{Alberts and Hayes} in~\cite{Alberts2006}, it is possible to identify four different types of networks, not related to hierarchy, that makes an approach of C2 in Information age militaries.

\begin{itemize}
    \item Fully connected: every entity is connected to every other, or there is an interaction among all of them;
    \item Random networks: each entity has the same probability of interacting with any other;
    \item Scale-free networks: a few entities have a very large number of connections or interactions with other entities;
    \item Small world networks: very high cluster coefficient, with a number of connections around $log(N)$, where $N$ is the total of nodes \cite{small01}.
\end{itemize}
 
The richest network structure is that mixed with different types in each level\cite{Alberts2006}. It should be the one with a scale-free in high level, the intermediate level composed by small world networks and at local level, the using of fully connected network structure.

\textit{Zughe and Sun} \cite{small02} proposed a solution to represent a Small-World network using a virtual ring topology. In that work, it is presented some aspects that permits a simpler approach to represent this kind off communication strucutre.

\textbf{c) Distribution of Information (DoI)}

How the information is shared, following the Information Exchange Requirements (IRE) and get the shared awareness. Based on this, it is presented a conceptual model to C2 Agility, that is the capability of C2 to successfully effect, cope with or exploit changes in circumstances.\cite{ABAR201713} 



%%%%%%%%%%%%%%%%%%%%%%%%%%%%%%%%%
%%%%%%%%%%%%%%%%%%%%%%%%%%%%%%%%%

\subsection {Network Enabled Capabilities}

The Network Enabled Capabilities - NCW (England) or Network Centric Warfare (USA) - NEC was created to increase the action power through a shared awareness that can be synchronized using a specific network structure and organization to create entities interaction \cite{Alberts2000}. 

Based on NEC paradigm, it was identified five archetypes to C2 Approach. These archetypes are listed in Table \ref{table:nec} and are defined as \gls{C2 Strategy} with their characteristics. Each \gls{C2 Strategy} adopted is related to a certain maturity level and occupies a specific position in C2 Approach Space.

\begin{table}[ht]
	\small
	\fontsize{6}{6}\selectfont
	\centering
	\caption{Network types associated to C2 Approaches}
	\label{table:nec}
	
	\begin{tabular}{p{0.1\linewidth}p{0.5\linewidth}p{0.2\linewidth}p{0.1\linewidth}}
	\hline
		\textbf{NEC Approach (C2 archetypes)}
		& \textbf{Archetype Characteristics}
		& \textbf{Network Type}
		& \textbf{Topology} \\ [1ex]
	\hline	
	Edge C2 & All elements are connected to everyone; There are $\frac{n(n-1)}{2}$ connections & Fully connected & Fully Connected \\[5ex]
	Collaborative C2 & Shared resources; entities interdependence; suitable to holistic problems that can not be fully decomposed; peer-to-peer communication & Scale-free Networks (a few nodes have a very large number of connections) & Star + Fully connected \\[5ex]
	Coordinated C2 & Suitable to problems that can be fully decomposed; share necessary information to execute what was defined & Random Networks (same probability to connect) & Star \\[5ex]
	De-Conflicted C2 & The situation can be decomposed with no cross impact; sub-optimized; not suitable to dynamic scenarios & Small World Networks ( $\log{N}$ connections & Ring \\[5ex]
	Conflicted C2 & No communication between entities & Isolated & - \\[1ex]
	\hline
	\end{tabular}
\end{table} 



The term Network Enabled Capability (NEC) looks for achieving the effect with the best usage of information systems. \textit{Alberts et al.} presented~\cite{nato01} the five NATO NEC C2 Maturity levels that represents a number of C2 Approaches to deal with different contexts as listed below.

\begin{itemize}
    \item Edge C2
    \item Collaborative C2
    \item coordinated C2
    \item De-Conflicted C2
    \item Conflicted C2
\end{itemize}

Each of these approaches corresponds to a specific region in the C2 Approach Space as shown in Figure \ref{c2s02}.

\figura[ht]{C2S_02}{NEC Approaches to C2 (from \cite{FRANCE2014})}{c2s02}{width=0.7\textwidth}%

These approaches are selected according to the mission and a set of circumstances. Different context requires specific \gls{C2} approaches. When some of these elements change, it is possible that some C2 approaching is not suitable anymore. The set of possible approaches available to be employed defines the Endeavor Space~\cite{FRANCE2014}.

SAS-060 \cite{nato01}, by \gls{NATO}, recommended following transition rules in case of endeavor complexity  changes showed in Table \ref{table:table02}.

\begin{table}[ht]
	\small
	\fontsize{10}{10}\selectfont
	\centering
	\caption{Endeavour complexity  changes}
	\label{table:table02}
	
	\begin{tabular}{ccccc}
	\hline
		\textbf{Endeavour Complexity}
		& \textbf{Appropriate C2 Approach} \\ [1ex]
	\hline	
	
	Low & De-conflicted \\[1ex]
	Medium & Coordinated \\[1ex]
	Medium-High & Collaborative \\[1ex]
	High - Very High & Edge \\[1ex]
	\hline
	\end{tabular}
\end{table} 

Communications improvements conduct the entities to a C2 Approach closer to the edge. This more connected structure becomes the awareness spread easier, due to the information sharing and collaboration among entities. The network structure and robustness facilitate to share awareness among dispersed entities. 


%%%%%%%%%%%%%%%%%%%%%%%%%%%%%%%%%
%%%%%%%%%%%%%%%%%%%%%%%%%%%%%%%%%

\subsection{Endeavor Space}

The set of all requirements, circumstances and conditions of a mission forms the endeavor space. It is created when all possible changes in circumstances that may cause impacts in an entity, and these possibilities are systematically identified and mapped ~\cite{FRANCE2014}. 

Precisely, the Endeavor Space is identified during planning time to try to preview the changes in circumstances as more precise as possible. It reduces the inherent uncertainties during mission execution~\cite{FRANCE2014}.

Each region of the endeavor space requires an option in C2 Approach Space to deal with the challenges and to perform the mission given. Different circumstances and other context changes implies in different regions in endeavor space.

A computational view of endeavor space is a search space to the C2 functions application. At different moments, these functions have a specific region of endeavor space as domain values, that conduct to the result or objective, e.g., mission accomplishment. The area formed by the sum of all regions where an entity operates defines the entity's agility in a general way \cite{Alberts2011}.

Based on the domain studies, these changes in circumstances that conduct to different endeavor space positions and these changes can be divided in three different areas: self, environment and mission~\cite{Alberts2006}.

%%%%%%%%%%%%%%%%%%%%%%%%%%%%%%%%%
%%%%%%%%%%%%%%%%%%%%%%%%%%%%%%%%%

\subsection {Functions of C2}

A C2 application permits the identification of some related functions that are required in order to apply C2 concepts and principles~\cite{Alberts2006}. These functions are listed bellow and are applicable to many endeavor types (military, civilian and industrial). They are related to C2 and describe the steps to be fellow:
\begin{itemize}
    \item Establishing intent
    \item Determining roles
    \item Establishing rules and constraints
    \item Monitoring and assessing the situation and the progress
\end{itemize}

However, it is necessary to consider the leadership factor. How the commanders or managers are good as leaders guiding the team to the mission or task accomplishment. To this aspect, it is necessary to add the following functions:

\begin{itemize}
    \item Inspiring, engendering trust and motivating
    \item Training and education
    \item provisioning
\end{itemize}

These three last functions are related to human aspects and behaviors. They depend on the actor that are as commander or manager.


%%%%%%%%%%%%%%%%%%%%%%%%%%%%%%%%%
%%%%%%%%%%%%%%%%%%%%%%%%%%%%%%%%%

\subsection {Roles and Responsibilities}

According to~\citet{Alberts2006}, C2 always refers to many individuals or entities. There is no sense to talk about C2 when we have only one member. And these entities or individuals present behaviors or roles that define a collaboration environment to determine the ability of missions accomplishment. Thus, the roles or responsibilities need to be defined separately and guarantee their interaction modelling.

The scenario involved can establish multiple roles to the same individual and roles scattered among more than one element, e.g., a specific role executed in two parts by two different members. Under the role perspective, we have a crosscutting by more than one member. In terms of the individual view, we can have roles tangled within it.

The rights exercised by each member are the result of their roles or responsibilities and define the Allocation of Decision Rights (ADR) axis in the C2 Approach Space~\citet{FRANCE2014}.

This capability impacts on the modelling way (roles scattered among different members)
