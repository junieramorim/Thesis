Agility is, in many scenarios, hard to be defined and many definitions can conduct us to a conflicting idea related simply to performance and fast action. Furthermore its definition can suffers slight changes according to the domain where it is been applied or measured~\cite{Gren2019AgilityIR}\cite{Alberts2011}.

In terms of software context, let's consider what was presented by \textit{Gren et al.}~\cite{Gren2019AgilityIR} that defines agility as the ability to create and respond to change in order to succeed in an uncertain and turbulent environment. This definition is aligned with what is applied in Software Engineer and adaptive systems architecture. A similar definition can be identified in another context as what was presented in NATO Report \textit{SAS-085}~\cite{FRANCE2014}. According to that document, agility is the capability to successfully effect, cope with and/or exploit changes in circumstances. 

Both definitions above are based on changes in circumstance due to an uncertainty in the conditions related to the system. The capability to deal with these changes and uncertainty indicates how agile is the entity, being a software or not.



\subsection{C2 Agility}
C2 Agility (NATO): ``C2 Agility is the capability of C2 to successfully effect, cope with, and/or exploit changes in circumstances''

Command and Control (NATO): ``a function, one that includes the allocation of decision rights across the enterprise, the shaping of enterprise decision-making processes and the processes that acquire, manage, share, and exploit information in support of individual and Collective decision making''
