a) Seção 1

Pergunta 1: O Cenário 1 apresentado utiliza-se de simulações de uso de VANTs através de software, onde há respostas a mudanças de contexto. O Sr concorda que este cenário é relevante  para a sua Organização/Unidade?  

Tipo de resposta: Escala de 1(discordo totalmente) a 5 (concordo totalmente)

Objetivo: Atender ao item 2.2 dos objetivos recebendo o retorno da real relevância do cenário proposto quando comparado com as situações enfrentadas pelo domínio.



Pergunta 2: Justifique a resposta anterior (máx. 15 linhas)

Tipo de resposta: Texto livre

Objetivo: Permitir ao especialista exprimir as motivações da resposta anterior de modo a identificarmos o grau de distanciamento do cenário simulado em relação às aplicações no mundo real.


Pergunta 3: O que poderia ser incluído/alterado no referido cenário de modo a torná-lo mais próximo de uma situação real de interesse para a sua Organização/Unidade?

Tipo de resposta: Texto livre

Objetivo: Permitir ao especialista listar o que poderia ser incluído e/ou alterado no cenário de simulação apresentado de modo que ele se torne mais fiel às condições reais vividas pela organização/unidade do entrevistado.


Pergunta 4: Sob seu ponto de vista e experiência, quais seriam as quantidades adequadas para cada elemento do Cenário 1 apresentado?

Tipo de resposta: Fechada com números pré-definidos para serem selecionados

Objetivo: Obter do especialista do domínio, quais serias as quantidades mais adequadas de elementos no cenário de modo a tornar-se mais aderente ao mundo real.


Pergunta 5: O slide 20 caracteriza o cenário 1 como sendo composto de 1 VANT executando os papeis de coordenador e distribuidor das tarefas, e os demais VANTs com a função de executor. Nesse contexto, o coordenador possui total autonomia (não depende de um comando central) e utiliza a Estratégia de C2 para a troca de informações com o restante do grupo/fração. Considerando a lista abaixo de possíveis respostas do sistema a eventuais mudanças de contexto, como o Sr classifica cada uma delas em termos de relevância/importância? (1 indica nenhuma relevância/importância e 5 indica totalmente relevante/importante)

Tipo de resposta: Escala de 1 a 5 para cada reação do sistema em resposta às mudanças de contexto.

Objetivo: Permitir que o especialista do domínio classifique, em grau de relevância, as atuais possíveis respostas implementadas no cenário simulado apresentado.


Pergunta 6: Revendo os slides e o vídeo que descrevem o Cenário 1, podemos afirmar que o sistema composto pelos VANTs é dotado de agilidade. Qual é o nível de concordância do Sr diante dessa afirmação? 

Tipo de resposta: Escala de 1(discordo totalmente) a 5 (concordo totalmente).

Objetivo: Identificar se, sob a visão do especialista, o cenário é capaz de deixar explícito as características de um sistema ágil aplicando estratégias de C2.


Pergunta 7: Considerando o Cenário 1 dos slides apresentados, quais dos recursos listados abaixo conferem agilidade de C2 ao time de VANTs ?


Tipo de resposta: Múltipla escolha com 6 alternativas que podem ser marcadas livremente. 

Objetivo: Identificar a capacidade do especialista em caracterizar o sistema ágil apresentado por meio do Cenário 1.




Seção 2

Pergunta 1: Descreva um cenário dinâmico planejado ou vivido pela sua Unidade/Organização com a aplicação dos princípios de C2. (Máximo de 30 linhas)

Tipo de resposta: Texto livre.

Objetivo: Permitir que o especialista descreva, de modo livre e com a linguagem do domínio, um cenário que aplique os princípios de C2 e que seja dinâmico.


Pergunta 2: Baseado na descrição das Estratégias de C2 apresentadas nos slides disponibilizados, qual delas está sendo operada pela Organização/Unidade no cenário descrito pelo Sr. ?

Tipo de resposta: Múltipla escolha com 6 alternativas onde apenas 1 pode ser marcada.

Objetivo: Verificar se o especialista do domínio identifica alguma das Estratégias de C2 apresentadas em um cenário real vivido pela própria organização.


Pergunta 3: Baseado no cenário dinâmico da sua Unidade/Organização descrito pelo Sr, quais das ações abaixo são aplicadas em resposta a uma mudança de contexto durante a execução da missão?

Tipo de resposta: Múltipla escolha com 9 alternativas que podem ser marcadas livremente.  

Objetivo: Permitir que o especialista identifique quais comportamentos esperados pelo cenário do domínio conferem resposta à possibilidade de mudanças de contexto.


Pergunta 4: No Cenário descrito pelo Sr., a Unidade/Fração é dotada de AGILIDADE?

Tipo de resposta: Múltipla escolha com 2 alternativas (SIM/NÃO) onde apenas 1 pode ser marcada.

Objetivo: Verificar se, sob o ponto de vista do especialista do domínio, o cenário real apresentado garante agilidade à sua Unidade/Fração.


Pergunta 5: Justifique a resposta anterior.

Tipo de resposta: Texto livre.

Objetivo: Entender o ponto de vista do especialista do domínio acerca da avaliação da sua própria Organização/Unidade em termos de nível de agilidade.


Pergunta 6: É fundamental o cumprimento total da missão, incluindo todas as tarefas que a compõe. Qual é o nível de concordância do Sr diante dessa afirmação? 

Tipo de resposta: Escala de 1(discordo totalmente) a 5 (concordo totalmente).

Objetivo: Identificar se, para o domínio, existe coerência e relevância na execução parcial de uma missão, permitindo que os elementos possam descartar as tarefas que por ventura não sejam realizáveis.


Pergunta 7: A missão deve sempre ser cumprida no menor tempo possível, em detrimento da qualidade de sua execução. Qual é o nível de concordância do Sr diante dessa afirmação?

Tipo de resposta: Escala de 1(discordo totalmente) a 5 (concordo totalmente).

Objetivo: Identificar a visão do especialista do domínio sobre a relação velocidade versus qualidade do cumprimento de uma missão.


Pergunta 8: Qual(is) papel(eis)/função(ções) podem ser identificados no cenário apresentado pelo Sr?

Tipo de resposta: Múltipla escolha com 4 alternativas que podem ser marcadas livremente.  

Objetivo: Verificar se o especialista do domínio identifica algum dos papéis definidos em nosso modelo no cenário apresentado por ele representando uma situação real.