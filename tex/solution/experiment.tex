
%Revisitar o Projeto do Estudo Empírico (Definir cenários).
%Projeto da simulação (Prof Travassos - Simular simulação)
%Design experimental.
%Protocolo de simulação.
%Liga GQM e Implementação (Justificar os cenarios)

%Explicar a simulação utilizando um sistema baseado em agentes (RePast) implementando o emprego de VANTs com sensores a bordo e empregados em uma missão de reconhecimento. 

% A alocação das tarefas que compoem a missão é realiada pelos próprios membros. Esses membros se coordenam através da abordagem de C2.

This study is classified as an empirical software engineering and we perform \textit{in silico} experiments~\cite{simulation01}, where the subjects and the world elements are described as computer models. Such method provides a way to analyse different situations and it allows to work with different scenarios that would be unfeasible to test given the cost and possibility of obtaining the resources.


The simulation implemented uses an agent-based technology to represent the entities, the tasks that they must perform and the environment where everything runs. This model is a simpler view of a real mission with use of UAVs. Each UAV is modelled as a DSPL able to configure itself according to the context requirements.

The simulated scenario composed by a set of members, i.e., a team, with a mission to be accomplished, i.e., a set of tasks, follows the steps below:

\begin{enumerate}
    \item The team selects a C2 Approach to be operated;
    \item The team performs the allocation tasks according to the information obtained through the C2 structure;
    \item The members receive the tasks allocated and performed the configuration required to execute them;
    \item The members goes to the task to be executed (displacement to the target);
    \item The member executes the task and reports the success accomplishment.
\end{enumerate}

However, these steps only works to a stable context. In our case, we have a dynamic scenario and with the purpose of evaluating the model, we submit the scenario to changes to simulate what could happen in the real world. Table \ref{table:context_changes} shows different circumstances explored by the simulation, with changes in the members, i.e., self, in the mission and in environment.

\begin{table}[ht]
	\small
	\fontsize{12}{12}\selectfont
	\centering
	\caption{Context changes handled by the simulator}
	\label{table:context_changes}
	
	\begin{tabular}{p{0.15\linewidth}p{0.8\linewidth}}
	\hline 
		 \textbf{Context Change}
		& \textbf{Description} \\ [1ex]
	\hline	
	Self & Sensors onboard damages caused by any internal issue (electronic circuit damaged)  \\[1ex]
	& UAV out of operation due to serious damage (e.g., no fuel or battery, or dropped) \\[5ex]
	
	Mission & tasks addition or removal in the mission\\[5ex]
	
	Environment & Weather conditions (e.g., luminosity, cloudiness) impacting the sensor's quality \\[1ex]
	& Hazard level requiring changes in the UAV's behaviour \\[1ex]
	\hline
	\end{tabular}
\end{table} 


Based on the members behaviour described by the PGs, Table \ref{table:scenarios} shows the main steps followed by the entities in a dynamic context.

% \usepackage[normalem]{ulem}
% \usepackage{tablefootnote}
% \usepackage{colortbl}


\begin{table}[h!]
\centering
\fontsize{11}{11}\selectfont
\label{table:scenarios}
\begin{tabular}{|l|l|l|} 
\hline
\multicolumn{3}{|c|}{{\cellcolor[rgb]{0.753,0.753,0.753}}\textbf{Self} }\\ 
\hline
\multicolumn{3}{|c|}{\textbf{Sensor } }\\ 
\hline
\begin{tabular}[c]{@{}l@{}}1. Select C2A\\2. Task allocation\\3. Member configuration\\4. Task execution\\\textbf{\uline{5. Sensor issue}}\\\textbf{\uline{6. Member reconfiguration}}\\7. Task execution\\8. Success report \end{tabular}                                             & \begin{tabular}[c]{@{}l@{}}1. Select C2A\\2. Task allocation\\3. Member configuration\\4. Task execution\\\textbf{\uline{5. Sensor issue}}\\\textbf{\uline{6. Task reallocation}}\\\textbf{\uline{7. Member configuration\tablefootnote{Come configuration can be required in order to deal with the new tasks received after reallocation;}}}\\8. Task execution\\9. Success report \end{tabular} & \begin{tabular}[c]{@{}l@{}}1. Select C2A\\2. Task allocation\\3. Member configuration\\4. Task execution\\\textbf{\uline{5. Sensor issue}}\\\textbf{\uline{6. C2A selection\tablefootnote{In case of no solution with a reallocation, the problem is escalated to the C2 Approach selector;}}}\\\textbf{\uline{7. Task allocation}}\\\textbf{\uline{8. Member configuration}}\\9. Task execution\\10. Success report \end{tabular}             \\ 
\hline
\multicolumn{3}{|c|}{\textbf{Member} }\\ 
\hline
\multicolumn{2}{|l|}{\begin{tabular}[c]{@{}l@{}}1. Select C2A\\2. Task allocation\\3. Member configuration\\4. Task execution\\\textbf{\uline{5. Member issue}}\tablefootnote{In case of a member loosing and its tasks will be reallocated;}\\\textbf{\uline{6. Task reallocation}}\\\textbf{\uline{7. Member configuration}}\\8. Task execution\\9. Success report \end{tabular}}& \begin{tabular}[c]{@{}l@{}}1. Select C2A\\2. Task allocation\\3. Member configuration\\4. Task execution\\\textbf{\uline{5. Member issue}}\tablefootnote{the loss of any member will be staggered to a new C2 Approach selection;}\\\textbf{\uline{6. C2A selection}}\\\textbf{\uline{7. Task allocation}}\\\textbf{\uline{8. Member configuration}}\\9. Task execution\\10. Success report \end{tabular}  \\ 
\hline
\rowcolor[rgb]{0.753,0.753,0.753} \multicolumn{1}{|c|}{\textbf{Mission} }& \multicolumn{2}{c|}{ \textbf{Environment}}\\ 
\hline
\multicolumn{1}{|c|}{\textbf{Add/Remove Task}}& \multicolumn{1}{c|}{\textbf{Weather}}   \multicolumn{1}{c|}{\textbf{Hazard}}\\ 
\hline
\begin{tabular}[c]{@{}l@{}}1. Select C2A\\2. Task allocation\\3. Member configuration\\4. Task execution\\\textbf{\uline{5. Tasks added/removed}}\\\textbf{\uline{6. Task reallocation}}\\\textbf{\uline{7. Member configuration}}\\8. Task execution\\9. Success report \end{tabular} & \multicolumn{2}{l|}{Assumption: the same possibilities of Self changes}\\
\hline
\end{tabular}
\end{table}

\subsection{Executions}

We performed a set of executions to analyse the entities behaviour and to collect results to be used as evidence of C2 agility existing. These executions were done with the following scenarios:

\begin{enumerate}
    \item A set of members operating a C2 Approach (Edge, Coordinated, De-Conflicted, Conflicted) and sensors' issues occur
    \item A set of members operating a C2 Approach (Edge, Coordinated, De-Conflicted, Conflicted) and environment changes occur
    \item A set of members operating a C2 Approach (Edge, Coordinated, De-Conflicted, Conflicted) and some members are randomly strike.
    
\end{enumerate}


