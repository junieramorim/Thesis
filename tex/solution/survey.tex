The validation of this study was carried out in the military field of application. The selection of such domain is based on the fact that its scenarios are naturally endowed with a high degree of dynamism, in addition to having, in its structure, the roles and functions well defined and required for the proper implementation and deployment of concepts related to C2. Another factor that contributed to the choice of this domain was the proximity of the study group and support received by the Brazilian Army.

Conceptually, the existing organizational structures in the military domain employed for the fulfillment of their missions related to the final activity have hierarchically organized elements and are compatible with the model proposed in this work. In addition, the need for these structures to respond to changing circumstances is a relevant factor for the public in question. 

To validate the model proposed and to collect more realistic scenarios from the domain, it was applied a questionnaire that is divided in two parts with the following objectives:

\begin{itemize}
    \item Evaluate dynamic scenarios, simulated by software, with application of C2;
    \item Identify possible realistic dynamic scenarios that present the concepts of C2 agility.
\end{itemize}

Support materials have been created that include a set of slides for knowledge leveling and a reference source for the respondent, as well as a video showing the execution of the simulator showing the use of UAVs in reconnaissance missions. 

To design and execute this questionnaire, we followed the activities presented by \textit{Pfleeger et al.}~\cite{survey01}  to survey process definition. The first step is to define measurable objectives, and in our case we have two listed bellow:

\begin{enumerate}
    \item Submit dynamic scenarios with application of C2, simulated by software, to military experts evaluation;
    \item Identify possible real military dynamic scenarios that present C2 agility concepts.
\end{enumerate}

The survey planning and scheduling were done according to the organizations availability and it was applied from June to July. The appropriate resources in terms of survey respondent were obtained through a previous contact with the organizations.

The elaboration of the questionnaire seeks its effectiveness in obtaining the necessary information for the validation of our study. To obtain that, it meets the three quality requirements:

\begin{enumerate}
    \item Tries to be resilient to opinion influences so as not to influence responses. Care is taken not to induce the respondent to a particular response by withholding valuable information.
    \item The proposed questionnaire is appropriate for the domain and its questions, in addition to being closely linked to the subjects of knowledge of the domain, present complexity compatible with the target audience.
    \item The allocation of resources for the application and analysis of the results is compatible with reality and does not require excessive time commitment and application of other resources by the respondents. The low cost is in line with the high value and interest of the subject on the agenda by the members of the domain.
\end{enumerate}

The design chosen for the questionnaire is the Descriptive, and Case Control sub-type, which makes use of the analysis of previous situations experienced by the respondent, in order to help explain the effects and aspects that are the object of the study. The presentation of simulated scenarios, in order to provide the elements of analysis, also explores the capacity of the respondent based on their previous experiences and knowledge.

The data was collected using the Google Forms Personal tool~\cite{Google2020}. The set formed by 15 (fifteen) questions was submitted to a group of 3 experts to validate the questions, as well as the type of question adopted. All questions are listed in Appendix \ref{app:survey}. After the questions analysis made by domain experts, we collected feedback and all suggestions was accepted and the survey was updated.

With the survey ready, the population was selected and the support material with the questionnaire was made available and a deadline of 15(fifteen) days was set to reply, after receiving the email.




\subsection{Population and Sample}

The general population used in this questionnaire will consist of the staff of the following Organs (to be confirmed): COTER, CDS, MD and Artillery*.
Of these organizations, we will consider as target population the total number of personnel that meet the criteria listed below:

\begin{itemize}
    \item Military personnel working on the documentation of doctrines related to the application of C2 concepts;
    \item Military personnel who work in the development of technological tools that help the application of the C2 concepts in the Organization's final activities;
    \item Military personnel of different hierarchical levels and with different functions within the planning process of operations using C2 concepts;
    \item Military personnel of different hierarchical levels and with different functions within the process of applying C2 in operations;
    \item Military personnel with a leading role in operational activities;
\end{itemize}

From this target population, according to the rules and authorizations of each Organ/Unit, the samples will be defined through a stratification (strata) following the following criteria:

\begin{itemize}
    \item Officers who plan actions using the C2 concept (COTER and MD)
    \item Officers exercising command actions in operations using C2 concepts (Artillery*)
    \item Military personnel who participate in modeling processes and development of tools used in C2 operations (CDS)
\end{itemize}

The size of the samples will depend on how the organ is functioning, and on the corresponding authorization of the managers directly involved or other particular requirements of the interviewed organization; According to the design of the questionnaire and the characteristics of the organizations from which we draw the population, we may adopt the sampling for convenience. This definition will be made after contact with the Organs/units involved.


