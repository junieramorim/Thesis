The domain representation requires a definition to the satisfiability of requirements using QPTL~\cite{qptl01}. A C2 system aims to perform a mission, characterized as a set of tasks $T$. This execution is done by a set of members $E$, i.e., a team, that receive the tasks allocated according to their capabilities and resources. This allocation occurs because there is a C2 Approach $\omega$ that guarantees a communication structure between the members.

All members $e_j$ of a team $E$ are modelled as a DSPL and represented by their Feature Model ($FM_j$). There are valid configurations in it that permits the software onboard to be tailored according to the tasks allocated in order to become the member compatible to the task execution. The software adaptation is performed in runtime according to what is obtained by the sensors about the member itself or about the environment. As the system must to provide agility, it must be able to present a response in case of context changes. It resumes that a member presents a configuration $c$ that satisfies a specific task $t$ allocated. This satisfability is shown by Equation \ref{sat03} and does not consider quality levels.

\begin{center}
\begin{equation}
\label{sat03}
c \models t \Longleftrightarrow \exists (F=[f_1, ...,f_k]) \in c \centerdot \ \exists (\mathcal{R}(F, t)),\ where\ k \geq 1
\end{equation}
\end{center}

where $\mathcal{R}$ represents a function (it can be a higher-order function) that correlates the list of features $F$ to the allocated task $t$. This modelling guarantees agility due to the adaptability tried during execution when changes in context occurs.

An example of this relation that links features and tasks is a table that indicates a member execution compatibility level with the sensor quality to perform the task. The features represent the sensor operated and it can be enabled or not according to requirements. Table~\ref{table_example01} shows this example where two different sensors onboard in a member are represented as features in its $FM$ with different levels of result when applied to perform different types of tasks. We can observe that VGA sensor is incompatible to perform the task $t_2$ and thermal sensor is fully suitable to the task $t_2$. These values are used during the allocation procedure to determine if the task $t$ will be included in the subset of tasks $T_j$ of the member $e_j$.

\begin{table}[ht]
	\centering
	\caption{An example of relation between \textit{features} and tasks}
	\label{table_example01}
\begin{tabular}{|l|l|l|l|}
\hline
\multicolumn{1}{|c|}{\textit{Feature}} & \multicolumn{1}{c|}{\textit{Description}} & \multicolumn{1}{c|}{$t_1$} & \multicolumn{1}{c|}{$t_2$} \\ \hline
$f_1$                                  & VGA Sensor                                & 0.65                       & 0.00                       \\ \hline
$f_2$                                  & Thermal Sensor                            & 0.30                       & 1.00                      \\ \hline
\end{tabular}
\end{table}


If a member $e_j$, as a FM, has a configuration $c$ that satisfies the task according to Equation \ref{sat03}, we can say that:

\begin{center}
\begin{equation}
\label{sat02}
e_j \models t \Longleftrightarrow \exists c \in \llbracket FM_j \rrbracket \ | \ c \models t
\end{equation}
\end{center}

The satisfiability of a task indicates that exists a valid configuration of a member that is compatible with this task and becomes a member able to perform it. 

As explained in Section~\ref{sec:context}, A C2 System always operates a C2 Approach ($C2_{ap}$) $\omega$ that belongs to the set of all identified C2 Approaches $\Omega$. If a member satisfies an allocated task, it means that the $C2_{ap}$ under which the member is operating satisfies that task too. Based on this, Equation~\ref{sat04} shows the $C2_{ap}$ satisfiability during system execution.

\begin{center}
\begin{equation}
\label{sat04}
\omega \models t \Longleftrightarrow  \Diamond \exists e_j \in E \centerdot  ((e_j \models t) \land (t \in T_j))
\end{equation}
\end{center}

Where $T_j$ is the set of tasks allocated to the member $e_j$.

The temporal operator, i.e., eventually $\Diamond$, indicates the dynamic context and the capability to adapt the system to the changes, i.e., all tasks will be allocated at some point in the future during the mission execution according to the context awareness obtained with the C2 Approach adoption.

Considering the CS operation, we can identify in the TA role that the task allocation is performed by the function $f\_alloc([t_1, t_2, ..., t_n] p1, [e_1, e_2, ..., e_n] p2)$ that receives a list of tasks and a list of members and makes the allocation. The return is a list of pairs $\langle T_i, i \rangle$ that maps a list of tasks $T_i$ for each member whose index is $i$. Thus, a task $t$ is allocated to the member $e_i \in E$ if $f\_alloc([t],E)=[\langle i, [t] \rangle]$. Based on this, we can rewrite Equation˜\ref{sat04} as

\begin{center}
\begin{equation}
\label{sat05}
\omega \models t \Longleftrightarrow   \Diamond \exists e_j \in E \centerdot ((e_j \models t) \land (TA.f\_alloc([t],E)=<j,[t]>))
\end{equation}
\end{center}

As $T_j$ is part of the whole mission, i.e., $T_j \subseteq T$, we can generalize the $C2_{ap}$ to all tasks with the Equation~\ref{sat05} considering the response capability to context changes. In that case, the C2 Approach may change according to conditions providing agility to the system and ensuring the correct task allocation.

\begin{center}
\begin{equation}
\label{sat05}
\omega \models T \Longleftrightarrow \forall t \in T \Diamond \exists \omega \in \Omega \centerdot \ \omega \models t 
\end{equation}
\end{center}

Since there is no collaboration among the members, i.e., each task can only be allocated to a member during any time up to its conclusion, each member $e_i$ gets an allocation set $T_i$. A complete allocation, i.e., all tasks of a mission are allocated to some member, the mission $T$ is composed by the all tasks allocated. Equation~\ref{sat06} shows this composition.

\begin{center}
\begin{equation}
\label{sat06}
\bigcup\limits ^{|E|}_{i=1} T_i \subseteq T
\end{equation}
\end{center}

To have $e_j \models t$ satisfied, the C2 Approach $\omega"$ under which that member is operating has to satisfy the task too, i.e., $\omega \models t$. Generalizing this concept to a set of tasks, i.e., the mission $T$, and a team $E$ of members operating a C2 Approach $\omega$, we can rewrite the Equations~\ref{sat02} and ~\ref{sat04} as,

\begin{center}
\begin{equation}
\label{sat01}
E \models_\omega T \Longleftrightarrow \Square \forall t \in T \Diamond (\exists \omega \in \Omega, \exists e_j \in E \ \centerdot \ ((\omega \models t) \land (e_j \models t)))
\end{equation}
\end{center}

The temporal constraints guarantee the system ability to deal with context changes in runtime. This capability provide an agility level to the C2 system. Thus, if Equation~\ref{sat01} is satisfied, our system have agility in any level.
