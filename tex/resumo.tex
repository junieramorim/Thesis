Comando e Controle (C2), em sua origem histórica, está ligado à aplicação de estratégias militares clássicas onde tínhamos um único comando centralizado e uma cadeia hierárquica inflexível entre os elementos que compunham as forças atuantes. C2 não é um fim por si mesmo, mas um processo cujo objetivo é otimizar a aplicação de recursos a fim de realizar uma dada missão. No entanto, em um contexto de C2 moderno, o dinamismo da missão, da equipe e do ambiente é um pressuposto necessário e, assim, a organização da equipe para realizar uma missão torna-se um desafio exigindo adaptações constantes. Essa capacidade de adaptação a novas circunstâncias caracteriza a Agilidade de C2. Entretanto, os estudos atuais não evidenciam os impactos das escolhas da estratégia de C2, representada pelo nível de espalhamento da informação, pela organização dos executantes e pela capacidade de tomada de decisão, nessa capacidade de agilidade. Além disso, os trabalhos recentes não consideram a medição de Atributos de Qualidade (QA), o que torna os modelos e simulações pouco aderentes à realidade das missões, onde ao menos o custo pode ser impeditivo à sua realização. Para tratar essas questões, o nosso objetivo é aplicar conceitos de Sistemas Auto-Adaptativos (SAS) com uma abordagem utilizando Linhas de Produto de Software Dinâmico (DSPL) para representar os elementos que compõem o sistema de C2 e que se organizam em times. Trabalhando com os conceitos de configuração e coordenação, propomos um processo a ser aplicado em C2 que prevê a geração de modelos que buscam garantir agilidade em tal contexto. Tais modelos preveem a escolha adequada da estratégia de C2 para a correta coordenação dos membros, aliada à capacidade de reconfiguração dos  executores a fim de garantir agilidade para enfrentar as mudanças de circunstância que podem ocorrer. Para avaliar a solução proposta, planejamos aplicar questionários e workshops a especialistas do domínio de C2, além de realizar simulações que evidenciem que as hipóteses traçadas estão corretas.