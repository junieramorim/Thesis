%%%%%%%%%%%%%%%%%%%%%%%%%%%%%%%%%%%%%%%%
% Classe do documento
%%%%%%%%%%%%%%%%%%%%%%%%%%%%%%%%%%%%%%%%

% Opções:
%  - Graduação: bacharelado|engenharia|licenciatura
%  - Pós-graduação: [qualificacao], mestrado|doutorado, ppca|ppginf

% \documentclass[engenharia]{UnB-CIC}%
\documentclass[qualificacao,doutorado,tese,ppginf]{UnB-CIC}%

\usepackage{pdfpages}% incluir PDFs, usado no apêndice
\usepackage{amsmath}%
\usepackage{amssymb}
\usepackage{MnSymbol}%
\usepackage{wasysym}%
\usepackage{color, colortbl}
\usepackage{glossaries}  
\usepackage{tikz}
\usepackage{stmaryrd}
\usepackage{eucal}

%For the tables
\usepackage[normalem]{ulem}
\usepackage{tablefootnote}
\usepackage{colortbl}

\usepackage{rotating}

\usepackage{indentfirst} %identa o primeiro paragrafo
\usepackage{tabularx}
%%%%%%%%%%%%%%%%%%%%%%%%%%%%%%%%%%%%%%%%
% Informações do Trabalho
%%%%%%%%%%%%%%%%%%%%%%%%%%%%%%%%%%%%%%%%
\orientador{\prof \dr Vander Alves}{CIC/UnB}%
\coorientador{\prof \dr Edson Pignaton de Freitas}{INF/UFRGS}%
%\coorientador{\prof \dr José Ralha}{CIC/UnB}
\coordenador[a]{\prof[a] \dr[a] Genaína Nunes Rodrigues}{CIC/UnB}%
\diamesano{6}{dezembro}{2019}%


\membrobanca{\prof[a] \dr[a] Ingrid Nunes}{INF/UFRGS}%
\membrobanca{\prof[a] \dr[a] Genaína Nunes Rodrigues}{CIC/UnB}%
% CIC - Computer Science Department

\autor{Junier Caminha}{Amorim}%

\titulo{A Product Line Based Method to Provide Command and Control Agility}%

\palavraschave{Software Product Line, Command and Control, agility, dynamic}%
\keywords{Software Product Line, Command and Control, agility, dynamic}%

%\newcommand{\unbcic}{\texttt{UnB-CIC}}%

%%%%%%%%%%%%%%%%%%%%%%%%%%%%%%%%%%%%%%%%
% Texto
%%%%%%%%%%%%%%%%%%%%%%%%%%%%%%%%%%%%%%%%

\begin{document}
    \capituloE{introduction}{Introduction}{
This chapter introduces this work by first describing the context (Section~\ref{sec:context}). Next, follow the research problems and corresponding motivations (Section~\ref{sec:problemMotivation}). Then a proposed solution is presented (Section~\ref{sec:solutionIntro}) as well as its planned evaluation (Section~\ref{sec:evalIntro}). Finally, we discuss the remainder structure of this work (Section~\ref{sec:structureWork}).
}

    \capituloE{background/background}{Background}{
This chapter presents all main theoretical knowledge that will be the basis to all discussion about the problems and the proposed solution.
    }%
    
    \capitulo{related}{Related Work}%
    
    \capitulo{assumptions/assumptions}{Assumptions and Restrictions}%
    
    \capituloE{development}{Solution and Evaluation}{
In this chapter, we first detail the addressed research problems, highlighting important aspects that are more valuable to the domain context (Section~\ref{sec:problem}). We present an appropriate modelling to trace essential requirements of these problems to the proposed solution, defining a process that provides agility to a C2 System (Section~\ref{sec:solution}). 
    }%
    
    \capitulo{next_steps}{Next Steps}%
    
    %\apendice{Apendice_Fichamento}{Fichamento de Artigo Científico}%
    %\anexo{Anexo1}{Documentação Original \unbcic\ (parcial)}%
    
\newglossaryentry{C2 Strategy}{name={C2 Strategy},description={It is the same of C2 Approach and defines a portion within the C2 Approach Space}}  

\newglossaryentry{context}{name={Context},description={Involves the circumstances, state of the entities and the environment }}

\newglossaryentry{agility}{name={Agility}, description={
Agility is the capability to successfully effect, cope with, and/or exploit changes in circumstances}}

\newglossaryentry{C2}{name={C2}, description={Command and Control abbreviation}}

\newglossaryentry{C2 Agility}{name={C2 Agility}, description={
Capability of C2 to successfully effect, cope with, and/or exploit changes in circumstances. It is the composition of C2 Approach Agility and C2 Maneuver Agility}}

\newglossaryentry{C2 Maneuver}{name={C2 Maneuver}, description={The ability to maneuver in the C2 approach space}}

\newglossaryentry{DoD}{name={DoD}, description={Department of Defense, United States of America}}

\newglossaryentry{IBAMA}{name={IBAMA}, description={Instituto Brasileiro do Meio Ambiente e dos Recursos Naturais Renováveis}}

\newglossaryentry{NATO}{name={NATO}, description={North Atlantic Treaty Organization is a organization formed by 29 countries around military and strategic common interests}}

    \printglossary[title={Glossary}]
\end{document}%